\chapter{Obstacles}
In this chapter we describe the biggest difficulties we encountered while implementing our project,
and how we resolved them.

\section{High Entry Threshold}
Kubernetes is a fairly young project, but already highly advanced and complex. It is constantly
improving and changing. It is written in Go --- a programming language none of
us had used before starting work on this project. We had to get a good grasp of Kubernetes’
high-level philosophy, as well as dive deep into the detailed design of its objects and the
lower-level internals of how they are managed and manipulated. We had to wade through large amounts
of documentation (some of it freshly written for beta features) before we could make educated
decisions about our project.

All of the above created a very high entry threshold. We managed to succeed by
first gaining a high level overview of Kubernetes, and then starting the coding stage as fast as
possible. The iterative, incremental approach to the problem allowed us to learn from our own mistakes
and adopt new concepts at a fast pace. It should also not be forgotten that we were getting
invaluable help from our Google mentor at every stage of our project implementation.

\section{Versioning}
As mentioned above, Kubernetes is a fairly young project, initially released in 2014. As a result,
almost all of its components are regularly updated and stabilized. In November 2017, when we
started working on this project, Kubernetes' stable release was version 1.8. By January 2018 it was
already in version 1.9 and on 24th April the 1.10 release was introduced~\cite{releases}. We had to adjust our
project to the constant changes in our toolkit. Many of the resources we were working on were
being moved between their alpha, beta and stable releases during the work on our project. The key to
success was to constantly follow the Kubernetes blog~\cite{blog} and keep track of the most important updates
on this topic.

\section{Dependency management}
Dependency management has for a long time been a big problem in Go. The main decision we had to make
was the choice of a tool for managing dependencies in our project. Our first choice was a tool called
dep~\cite{dep} as recommended by most Kubernetes articles and because it seemed to be the most up-to-date one,
being the official experimental dependency manager. Unfortunately, at the beginning of our project,
dep was not yet production ready, and it didn’t meet our requirements. We encountered trouble adding
some dependencies and were trapped in dependency cycles that dep couldn’t handle gracefully. The
second manager we tried was godep~\cite{godep}. This tool is the predecessor of dep and was available in a more
stable version. Despite its lack of support for constraints on vendor versions, it fulfilled its
role for a few months. Trying to keep up with the latest changes in the Kubernetes we were trapped
in a cycle of manually updating the dependencies and adding them to the vendor directory as godep didn’t
support the dependencies not directly referenced in code. Another issue with godep was resolving the
subdependencies required by more than one dependency. As the stable version of dep had already been
released at that point, we decided to change back and regain the full flexibility of vendor
versioning. It seems to solve our dependency problems for now.

In order to provide users with the correct dependencies while making use of our project we decided
to keep the \texttt{vendor} folder in the project repository. This folder contains all the
dependencies, flattened and correctly versioned, i.e. compatible with our project. This is standard
practice in the Go ecosystem.

\section{Continuous Integration}
Due to our familiarity with Jenkins it was our first choice when setting up continuous integration.
As the project was meant to be open-source we decided to use Blue Ocean~\cite{blueocean} with pipelines plugin to
allow contributors to tweak the build process. The Jenkins setup allowed for unit testing and
linter checks. However, the outdated kernel on the server we had access to wasn’t capable of handling
Docker and any form of Kubernetes cluster --- essential for integration tests. Along with the
necessity to maintain the Jenkins server ourselves at the expense of the time we could have spent
on the project we decided to move the continuous integration to a hosted service.

With Travis-CI we were able to set up the whole environment we had on Jenkins along with the
integration tests setup we were previously lacking.

We had difficulties setting up a Kubernetes cluster without direct access to the machines of our CI.
In the end we had to ask Travis-CI
administrators for such privileges. We were able to identify the issue --- our application, when
built on Ubuntu Linux (16.04 used by Travis-CI), would link to the standard libraries, not present
on Alpine Linux used as a Docker image base.

