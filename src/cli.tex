\chapter{Command Line Tool}
In this chapter we describe \texttt{msp} --- a command line tool we created 
to facilitate usage of the MySQL Operator. First, we explain the basic CLI 
concept, followed by a description of technologies used, and finally a 
detailed description of the functionalities provided by our tool. We also 
provide a few usage examples to outline the most common use cases.

\section{CLI Basic}
\texttt{msp} provides a layer of abstraction over management of MySQLCluster 
and MySQLBackup objects. Rather than writing verbose yaml files and manually 
modifying Kubernetes’ configuration via \texttt{kubectl}, a user can easily 
perform most actions related to managing their MySQL cluster with a single 
`msp` command. Additionally, a CLI tool like this helps inexperienced users 
get started with our system, and allows Kubernetes system administrators to 
write more concise and expressive automation scripts.

\section{Technologies}
Like the rest of our project, \texttt{msp} is written in Go. Our implementation 
bases on Cobra~\cite{cobra} --- a framework for writing command line interfaces 
similar to many UNIX shell programs. Cobra is widely used in the Go and 
Kubernetes ecosystems, most prominently in \texttt{kubectl} itself. The 
library is designed to facilitate the creation of easily extensible and 
customizable CLI tools. Cobra programs are typically called with a list 
of structured arguments specifying an object, an action to perform on the 
selected object, and optionally a list of flags that customize the program’s 
behavior. 

\section{Functionalities}
The treelike command structure implies the natural order of the whole CLI 
tool. Consequently, each command in our tool has the following form: \\ 
\centerline{\textit{msp resource-type action resource-name flags}}
\texttt{msp} can be used to create, update, or delete both \texttt{MySQLClusters}
and \texttt{MySQLBackupSchedules}. Flags can be global or local, referring 
to resource types and actions. To set global options, needed for proper 
functioning of cli, \texttt{msp} uses the following flags:
\begin{itemize}
	\item Kubeconfig --- location of kubeconfig file needed to establish a network connection with Kubernetes cluster
	\item Namespace --- Kubernetes namespace in which resources will be created
	\item Force --- to ignore errors during cascade deleting
\end{itemize}


\section{MySQL Operator Use-cases}

These use cases show most of the actions that can be performed to manage a cluster and its data.

\subsection{Cluster}

\subsubsection*{Create}
\noindent Creating a new cluster with an already exisiting Secret.

\begin{lstlisting}
> msp cluster create "my-cluster" --secret "my-secret"
\end{lstlisting}

\noindent Creating a new cluster along with defining new Secret.

\begin{lstlisting}
> msp cluster create "my-cluster" --secret "my-secret" \
	--password "password"
\end{lstlisting}

\noindent Creating a new cluster from a backup instance.

\begin{lstlisting}
> msp cluster create "my-cluster" --backup "backup-name" \
	--instance "backup-instance"
\end{lstlisting}

\noindent Creating a new cluster with specified port, image, storage and number of replicas.

\begin{lstlisting}
> msp cluster create "my-cluster" --port 1337 --image mysql:img \
	--replicas 7 --secret "my-own-secret"
\end{lstlisting}

\subsubsection*{Delete}

Command line interface provides users with more functionalities. Using 
\texttt{msp}, users can perform a hard delete --- remove not only the 
cluster but also all database information.

\noindent Deleting two clusters.

\begin{lstlisting}
> msp delete cluster "my-cluster" "my-cluster2"
\end{lstlisting}

\noindent Deleting cluster along with PVC’s.

\begin{lstlisting}
> msp delete cluster "my-cluster" --remove-pvc
\end{lstlisting}

\subsubsection*{Update}
\noindent Updating cluster with new secret, port and number of replicas.

\begin{lstlisting}
> msp update cluster "my-cluster" --replicas 4 --port 1337 \
	--secret "my-new-secret"
\end{lstlisting}

\subsection{Backup}

\subsubsection*{Create}
\noindent Creating a backup schedule that creates a backup instance at 23:59 December 31 every year.

\begin{lstlisting}
> msp create backup --cluster "my-cluster" \
	--time "59 23 31 DEC Fri *"
\end{lstlisting}

\noindent Creating a backup schedule named "elite" in Kubernetes that creates 
a backup instance at 13:37 every day. The 2Gi of storage is shared among
the backup instances.

\begin{lstlisting}
> msp create backup --name "elite" --cluster "my-cluster" \
	--time "13 37  *  *  * *" --storage 2Gi
\end{lstlisting}

\subsubsection*{Delete}
\noindent Deleting two backup schedules.

\begin{lstlisting}
> msp delete backup "my-backup" "my-backup2"
\end{lstlisting}

\noindent Deleting backup schedule along with PVC's.

\begin{lstlisting}
> msp delete backup "my-backup" --remove-pvc
\end{lstlisting}

\subsubsection*{Update}
\noindent Updating a backup schedule's time field.

\begin{lstlisting}
> msp update backup "my-backup" --time "12 20 * * * *"
\end{lstlisting}
