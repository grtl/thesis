\chapter{Final thoughts}
In this chapter we present our final thoughts and conclusions. We explain the possible extensions
and improvements to this project, which were not implemented due to time limitations and certain
project decisions.

\section{Future Development}
The project in its current form is fully functioning and ready to use. Nevertheless, there are many
ways of improving and extending the existing implementation. We will present a few examples.

The main improvement would be to secure the connections to the Kubernetes nodes. At the moment we
can access each node via an ncat connection. This was our simple solution to enable replication and
the creation of backups. However, this connection should be secured with a password.

Another good extension would be making use of Kubernetes events. Events are meant to provide
human-readable descriptions of actions related to this resource that have taken place. We did not
manage to implement it because events support for Custom Resources was not yet available during our
work on the project.

The final and biggest potential development is providing direct support for Apache/PHP applications.
For example, a \texttt{LAMPApplication} custom resource could be provided. The operator for these objects
would manage a replicated Apache deployment of a PHP app, along with a MySQL server cluster.
This could be handled in several ways --- either creating a MySQLCluster object and using the
operator for MySQL servers we already developed, or requiring the \texttt{LAMPApplication} spec to contain
the name of a service exposing a MySQL server. The latter solution would give more flexibility to
the user, but would require more steps to set up a working LAMP stack.

Additionally, this approach would only work for the simplest kinds of PHP applications.
Many commonly used PHP applications (Wordpress, Drupal, Joomla etc.) are not entirely stateless and
would require more intricate management. It would be impossible to design a \texttt{LAMPApplication} object
general enough to handle all of these cases. For the simplest case on the other hand user would not
benefit from adding another layer of abstraction, as they can be easily deployed as a single
Kubernetes Deployment.

\section{Conclusions}
The purpose of this project was to implement a LAMP stack for Kubernetes. We managed to complete
this task, with respect to a few project decisions. The Apache and PHP can be implemented as
deployments on docker images provided by users. Consequently, we focused on implementing the most
complex layer of LAMP --- a MySQL operator to facilitate the creation of a cluster of replicated
MySQL databases. Users are also provided with a CLI that allows for user-friendly interactions with
the operator.

Our project is fully open-source, released under the Apache License 2.0, just as Kubernetes is. As
far as we can tell, there is a demand for tools managing MySQL on this system. That is why our
project has such a wide appeal. At the moment of finishing work, it had 15 stars, 8 watches and 3
forks on Github. As for our competitors: at the beginning of 2018, Oracle started implementing their
own version of a MySQL Operator, available on Github~\cite{oracle}
