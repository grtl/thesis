\chapter{Introduction}
In the fast-paced world of modern web technologies, services are expected to
efficiently and reliably process large amounts of data. As such, developers
and system administrators want to create systems that are scalable. Scalability,
while desirable, is an issue that significantly adds to the complexity of
problems faced by software engineers every day. This has led to the emergence of
technologies that orchestrate entire clusters, providing automatization that
makes managing such systems easier.

\textbf{Kubernetes} is an open-source container orchestrator developed by
Google. It is used by leading tech companies to automate the deployment,
scaling, and management of their applications~\cite{kube-usecase}. Its common
use cases are cloud computing systems and microservices-oriented architectures.

The main feature provided by Kubernetes and similar orchestration
systems is the management of containerized applications. Containers are
a~lightweight and portable alternative to virtual machines. In recent years,
the~advantages of containerization have been recognised, which has led to
the~rapid development of supporting technologies, such as Docker\footnote{https://www.docker.com}
or rkt\footnote{https://coreos.com/rkt/}.

A central part of many web applications is a~performant database. Despite
the~variety of new database engines constantly appearing on the market, many
developers still decide to rely on \textbf{MySQL}. According to the db-engines
statistics~\cite{db-eng}, MySQL is the second most used database
engine.

Our project is focused on developing tools for setting up and managing
a~\textbf{LAMP}\footnote{Linux host with Apache web server, MySQL database,
and PHP as the server-side programming language.} stack in Kubernetes. Our main goal is to
extend the Kubernetes API with options for MySQL database
management and provide users with a user-friendly command line tool. As
a~result, developers should be able to more effortlessly benefit from
Kubernetes' advantages, such as scalability. Leveraging the powerful
abstraction system provided by Kubernetes in such a way will remove
much of the complexity traditionally caused by the database layer of
an~application.

There are existing unofficial tools with similar functionalities for various
databases and systems. Vitess~\cite{vitess} is an~open source
clustering system for MySQL databases that can interface with
Kubernetes. We are aiming for a simpler solution that utilizes
Kubernetes' native method of extending its API ---
\textbf{CustomResourceDefinitions}. Another existing tool,
the Crunchy Data PostgreSQL Operator~\cite{psql-op}, is closer in scope to what we want to
accomplish, providing a CustomResourceDefinition for Postgres databases.

This dissertation is divided into eight chapters. We start with the necessary
preliminary introduction to Kubernetes and custom resources. Next, we cover
our workflow and development stack. The main chapter describes the design and
implementation details of the MySQL Operator's functionalities. The following
chapter covers the command line interface we  provided to facilitate the usage
of our tools. The final sections explain main  design decisions, obstacles encountered,
and future considerations for further work on the project.

We collaborated closely as a team on both the MySQL Operator as well as this
dissertation. All design decisions were discussed between the four of us and
agreed upon by consensus. Marcin Chrzanowski was mostly responsible for
implementing the MySQLCluster operator's functionalities. Jakub Sarzyński
created most of the command line interface. Agnieszka Świetlik was responsible
for implementing the backup system. Mikołaj Walczak took on an unofficial tech
lead role and was also in charge of setting up our development infrastructure.
He administrated the Continuous Integration system, managed software
dependencies, and set up testing, in addition to contributing code to various
areas of the project.

Additionally, we would like to thank our Google mentor, Daniel Kłobuszewski. His
guidance was especially helpful during early stages of the project while we were
learning about Kubernetes. He also validated our design decisions.
