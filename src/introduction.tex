\chapter{Introduction}
In the fast-paced world of modern web technologies, services are expected to
efficiently and reliably process large amounts of data. As such, developers
and system administrators want to create systems that are scalable. Scalability,
while desirable, is an issue that significantly adds to the complexity of
problems faced by software engineers every day. This has led to the emergence of
technologies that orchestrate entire clusters, providing automatization that
makes managing such systems easier.

\textit{Kubernetes} is an open-source container orchestrator developed by
Google. It is used by leading tech companies to automate the deployment,
scaling, and management of their applications~\cite{kube-usecase}. Its common
use cases are cloud computing systems and microservices-oriented architectures.

The main feature provided by \textit{Kubernetes} and similar orchestration
systems is the management of containerized applications. Containers are
a~lightweight and portable alternative to virtual machines. In recent years,
the~advantages of containerization have been recognised, which has led to
the~rapid development of supporting technologies, such as Docker or rkt.

A central part of many web applications is a~performant database. Despite
the~variety of new database engines constantly appearing on the market, many
developers still decide to rely on \textit{MySQL}. According to the db-engines
statistics~\cite{db-eng}, \textit{MySQL} is the second most used database
engine.

Our project is focused on developing tools for creating and managing
a~\textbf{LAMP}\footnote{Linux host with Apache web server, MySQL database
and PHP programming language} stack in \textit{Kubernetes}. Our main goal is to
extend the \textit{Kubernetes API} with options for \textit{MySQL} database
management and provide users with a user-friendly command line tool. As
a~result, developers should be able to more effortlessly benefit from
\textit{Kubernetes’} advantages, such as scalability. Leveraging the powerful
abstraction system provided by \textit{Kubernetes} in such a way will remove
much of the complexity traditionally caused by the database layer of
an~application.

There are existing unofficial tools with similar functionalities for various
databases and systems. \textit{Vitess}~\cite{vitess} is an~open source
clustering system for \textit{MySQL} databases that can interface with
\textit{Kubernetes}. We are aiming for a simpler solution that utilizes
\textit{Kubernetes'} native method of extending its API ---
\textbf{Custom Resource Definitions} (CRDs). Another existing tool,
\textit{Postgres operator}~\cite{psql-op}, is closer in scope to what we want to
accomplish, providing a CRD for \textit{Postgres} databases instead of
\textit{MySQL}.

This dissertation is divided into four basic sections. We start with definitions
and an explanation of ideas used in the document. Next, we cover the necessary
preliminary knowledge about \textit{Kubernetes} and Custom Resources. The main
chapter describes our design decisions, caveats, and implementation details of
the~\textbf{LAMP} stack functionalities. The final sections explains the usage
of our program, and future considerations for further work on the project are
mentioned.
