\chapter{Development Stack}

In this chapter we provide the general overview of our workflow and
development stack.

\section{Version Control System}
Our version control system of choice is Git and we host our repository on
GitHub~\footnote{https://github.com}. Being the most popular git host,
GitHub is well integrated  with many developer tools such as Golang’s
\texttt{go get} and various continuous integration systems. On top of that,
we use common GitHub features for managing current tasks and project development.

\subsection{Workflow}
We follow a fairly standard git workflow. New features are developed on 
separate branches, branched directly off of \texttt{master}. Once the code
is ready, the branch is pushed to our central GitHub repository and code 
review is requested from other team members. After positive feedback from 
the CI system and two approvals, the pull request is merged to the 
\texttt{master} branch. To retain a linear git history, all feature branches 
are rebased onto the \texttt{master} branch before merging.

\section{Continuous integration}

\subsection{Travis}
Travis CI~\cite{travis} is a hosted continuous integration service with
a free option for open source projects.
Unlike many other services, it does not require server-side 
configuration, but relies only on .travis.yml files placed inside 
the project's source directory. This makes it easy to have different test 
code per branch and removes the necessity to manage permissions separately 
for the repository and for the CI system. Travis integrates well with other 
tools used in our project (GitHub and Slack).

Along with Travis CI, we have 
decided to use Coveralls~\cite{coveralls} --- a web service that 
tracks code coverage\footnote{the percentage of the source code of a~program 
is executed when a particular test suite runs} over time, and ensures that 
all new code is fully covered by tests.

\subsection{Kubernetes Cluster}
To perform integration tests during the continuous integration process we 
used Minikube~\cite{minikube} --- a lightweight
Kubernetes implementation that creates a VM on the local machine and deploys 
a simple cluster containing only one node.

\subsection{Continuous Deployment}
The Docker image built from the \texttt{master} branch is automatically 
deployed to DockerHub~\cite{docker}. This way users can download
and use our project easily as a docker image. Each stable version of our 
project is manually added and reflects a specific git commit.

\section{Communication}
As our communication tool, we have chosen Slack~\cite{slack} --- 
a modern IRC-like communicator. Slack features persistent chat rooms
(channels) organized by topic, comes with a large number of third-party 
services, and supports community-built integrations.
