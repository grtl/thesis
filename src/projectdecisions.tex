\chapter{Project decisions}
In this chapter we describe non-trivial decisions made while designing the project, as well as the
reasons behind them.

\section{Backup}
Even though it is certain that we want to use Custom Resource Definitions, the question on how to
use them still remains. The main object in the project is MySQL Cluster, which can be added,
deleted, updated, scaled freely as a CRD instance. The remaining question is about backups.

We decided to use the operator pattern again for managing backups. In the case of the MySQL
operator it was fairly obvious what needed to be done to implement a cluster --- create a
StatefulSet to run servers, and a Service to expose them at one DNS endpoint. Designing a system
for backups required a bit more thought.

\section{Cyclic Backups}
The Kubernetes’ design philosophy proclaims that users should describe the desired state and
Kubernetes will try to adjust to this request. As mentioned in the Background chapter, Kubernetes 
objects are by convention meant to represent nouns, not verbs. Following this ideology, we 
decided to provide cyclic backup creation. What is more, creating backups on a cyclic schedule 
is a good system administration practice. However, for the user’s convenience, one-shot backups are 
also supported.

\section{Backup Schedules and Instances}
In order to implement cyclic backups, there must exist objects responsible for managing each one of
these repetitive actions. At the same time, we have to keep track of existing backups. Taking this
into account, we provide users with the MySQLBackupSchedule custom resource, whose meaning is not a
verb ("create a backup"), but a noun ("an agent that creates backups periodically"). At the same
time, another resource --- MySQLBackupInstance is created to represent invdividual backup instances.

\section{LAMP Stack}
Our project’s purpose was to provide a convenient way to set up a LAMP
stack on Kubernetes. We quickly realized that the only part of this stack that needed special
consideration was the database layer. An Apache server serving a simple PHP web application can
easily be deployed using a Kubernetes Deployment and a publicly available Apache Docker image.
Kubernetes Deployment objects were designed specifically for these kinds of stateless applications.
As such, we decided to focus our efforts on implementing the MySQL operator.
